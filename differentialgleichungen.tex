\section{Differentialgleichung (DGL)}
Lineare DGL haben die allgemeine Form:
\[
	y^{(n)} + p_{n-1}(x) \cdot y^{(n-1)} + ... + p_1(x) \cdot y' + p_0(x) \cdot y = q(x) 
\]

\begin{description}
	\item [$y$] steht für $y(x)$ eine noch unbekannte Funktion von x.\\
			$y^{(i)}$ ist einfach die $i$-te Ableitung davon.

	\item [$p_i(x)$] steht für irgendeine Funktion mit der $y$ (oder $y^{(i)}$) multipliziert wird. Kann auch eine Konstante sein (z.B. 1).

	\item [$q(x)$] nennt man Störfunktion. Ist $q(x) = 0$ nennt man die DGL Homogen, sonst Inhomogen.
\end{description}

Die allgemeine Lösung einer DGL ist gegeben durch:
\[
	y(x) = y_h(x) + y_p(x)
\]
$y_h(x)$ ist die allgemeine Lösung der Homogenen DGL und\\
$y_p(x)$ ist die partikuläre Lösung der Inhomogenen DGL.

\subsection{Lineare DGL 1. Ordnung}
Diese DGL haben die allgemeine Form: $y' + p(x) \cdot y = q(x)$

\subsubsection{Lineare DGL 1. Ordnung mit konst. Koeffizienten}
Diese DGL hat die Form: $y' + a \cdot y = q(x)$ mit $a \in \R$\\

Vorgehen: Gleich wie im Fall von n, einfach mit n = 1.

\subsubsection{Lineare DGL 1. Ordnung mit var. Koeffizienten}
Diese DGL hat die Form: $y' + p(x) \cdot y = q(x)$\\
%Nach http://de.wikipedia.org/wiki/Variation_der_Konstanten#Motivation
Sei:\[
F(x) = \int f(t) dt.
\]
Dann ist $\{y_{Hom}(x) = c_1e^{F(x)}| c_1 \in \R\}$ die Menge aller Lösungen der homogenen Differnentialgleichung ($y' + f(x) \cdot y = 0$). Als Ansatz für die Lösung des inhomogenen Problems setze man $y_p(x) = u(x)e^{F(x)}$, d.h. man \textit{lässt die Konstante $c_1$ variieren}. Dies ergibt eine eindeutige Zuordnung zwischen den Funktion $y$ und $u$. Denn $e^{F(x)}$ ist eine stets positive, stetig differnezierbare Funktion. Die Ableitung dieser Ansatzfunktion ist \[
y_p'(x) = u(x)f(x)e^{F(x)} + u'(x)e^{F(x)} = y(x)f(x) + u'(x)e^{F(x)}
\]
Also löst $y$ die inhomogene Differntialgleichung \[
y_p'(x) = y(x)f(x) + g(x)
\]
genau dann, wenn\[
u'(x) = y(x)f(x) + g(x)
\]
gilt. Also folgt\[
u(x) = \int  g(t)e^{-F(t)} dt
\]
Somit ist die Lösungsmenge von $y_p$
\[
\{ y_p'(x) = e^{F(x)} (u(x) + c_2) | c_2 \in \R \}
\]
Die Lösungsmenge der generellen Lösung der allgemeinen Lösung ist somit $y$:
\begin{align*}
&\{y\}=\{y = y_{Hom} + y_p\}\\
& =\{ y(x) =  c_1e^{F(x)} +  e^{F(x)} (u(x) + c_2) | c_1, c_2 \in \R\} 
\end{align*}
Gibt es nun einen Ansatz, kann diese menge durch Einsetzen des Funktionswert und Gleichsetzen mit dem Resultat genau bestummen werde in dem die Konstanten $c$  aufgelöst werden.\\
\\
Konkret reicht es also aus. $F$ und $u$ zu berechnen. 

\subsubsection{Separierbare DGL}
Ist ein Spezialfall wo $q(x) = 0$ ist.
Eine Differentialgleichung für die Funktion $y$ heisst separierbar, wenn sie auf diese Form gebracht werden kann.
\[
	y' = p(x) \cdot y \hspace{1cm} \text{Merke: $y' = \frac{dy}{dx}$}
\]
Vorgehen:
\begin{enumerate}
	\item DGL auf obige Form bringen.

	\item $y'$ mit $\frac{dy}{dx}$ ersetzen und Variabeln trennen: $\frac{1}{y} \, dy = p(x) \, dx$

	\item Beidseitig integrieren: $\int \frac{1}{y} \, dy = \int p(x) \, dx$

	\item Nach $y$ auflösen

	\item Lösung: $y = C \cdot e^{\int p(x) \, dx}$ \hspace{0.5cm} $(C \in \R)$
\end{enumerate}
\textbf{Merke:} Weitere Lösungen kann die Gleichung $y = 0$ liefern, muss es aber nicht. Diese Lösungen sind konstanten (also $\in \R$).

\textbf{Beispiel:} $y' = \cos(x) \cdot y$
\begin{align*}
	\frac{dy}{dx} = \cos(x) \cdot y \Leftrightarrow \frac{1}{y} \, dy &= \cos(x) \, dx\\
	\int \frac{1}{y} \, dy &= \int \cos(x) \, dx \\
	\ln(y) &= \sin(x) + \ln(C)\\
	y &= e^{\sin(x) + \ln(C)} \\
	 &= C \cdot e^{\sin(x)}	\hspace{1cm} (C \in \R)
\end{align*}
{\small Allgemeiner Hinweis:\\
Bei logarithmischer Terme, wird Integrationskonstante zweckgemäss $\ln(C)$ gewählt (sonst müsste man später neue Konstante $C' = e^C$ einführen). Beachte das $y = 0$ auch eine Lösung ist!}



\subsubsection{genereller Ansatz}
Wenn $g(x) = 0$ ist, dann ist die DGL homogen. Falls $g(x) \neq 0$, so handelt
es sich um eine inhomogene DGL.

Der erste Schritt für homogene und inhomogene DGL ist die Lösung der homogenen
DGL: $y' + f(x) \cdot y = 0$:
{\small
\begin{align*}
y' + f(x) \cdot y &= 0 \quad \left | -(f(x) \cdot y) \right.\\
y' &= -f(x) \cdot y \quad \boxed{y' \text{ ist das gleiche wie } \frac{dy}{dx}}\\
\frac{dy}{dx} &= -f(x) \cdot y \quad \left | \div y \right.\\
\frac{dy}{dx\, y} &= -f(x) \quad \left | \int \right.\\
\int \frac{dy}{dx\, y} dx &= \int -f(x) dx \quad \boxed{\frac{dy}{dx\, y} \cdot dx = \frac{dx}{dx\, y} dy = \frac{1}{y} dy}\\
\int \frac{1}{y} dy &= \int -f(x) dx\\
\ln(y) &= -F(x) \quad \left | e^\alpha \right.\\
e^{\ln(y)} &= e^{-F(x)}\\
y &= e^{-F(x)}
\end{align*}
}

Damit erhalten wir die allgemeine Lösung: $y = A \cdot e^{-F(x)}$. Hat man eine
homogene DGL und einen Punkt, an dem die ursprüngliche Funktion ausgewertet wurde,
so kann man die explizite Lösung berechnen (also $A$ berechnen), in dem man die
hier allgemein erhaltene Lösung für den gegebenen Punkt auswertet und so die
Unbekannte bekommt.

Für ein inhomogenes DGL setzt sich die allgemeine Lösung aus der homogenen Lösung
$y_h$ und der partikulären (speziellen) Lösung $y_p$ der inhomogenen DGL zusammen.
Die homogene Lösung haben wir bereits berechnet: $y_h = A \cdot e^{-F(x)}$. Nun
folgt die partikuläre Lösung:

Dazu wird die Konstante ($A$) der homogenen Lösung als Funktion dargestellt ($u(x)$).
Wir erhalten somit: $y_p = u(x) \cdot e^{-F(x)}$.
Dieses $y_p$ setzten wir nun als $y$ in die inhomogene Gleichung ein:
{\small
\[
y' + f(x) \cdot y = g(x) \Rightarrow (\underbrace{u(x) \cdot e^{-F(x)}}_{= y_p = y})'
+ f(x) \cdot (\underbrace{u(x) \cdot e^{-F(x)}}_{= y_p = y}) = g(x)
\]
}

Die neue Gleichung wird nun nach $u'(x) = \ldots$ aufgelöst, was zu
$u'(x) = \frac{g(x)}{e^{-F(x)}}$ führt. Nun wird $u(x)$ bestimmt durch integrieren
beider Seiten: $u(x) = \int \frac{g(x)}{e^{-F(x)}}\,dx$. Hat man dies ausgerechnet,
setzt man $u(x)$ in $y_p = u(x) \cdot e^{-F(x)}$ ein und bekommt so die partikuläre
Lösung der DGL.

Als letzter Schritt für inhomogene DGL summiert man $y_h$ und $y_p$ und erhält nach
dem Umformen und Kürzen die allgemeine Lösung der DGL:
{\small
\[
y = y_h + y_p = 
\underbrace{A \cdot e^{-F(x)}}_{= y_h} +
\underbrace{\underbrace{\int \frac{g(x)}{e^{-F(x)}}\,dx}_{= u(x)} \cdot e^{-F(x)}}_{= y_p}
\]
}

Hat man für die inhomogene DGL ebenfalls Punkte an denen die Funktion ausgewertet wurde,
so kann man dies in die allgemeine Lösung eintragen und so die Unbekannten ($A$) berechnen.

\subsubsection{Beispiel mit Variation der Konstanten}
%Nach http://de.wikipedia.org/wiki/Variation_der_Konstanten#Motivation
Gegeben: $y' + x^2 \cdot y = 2x^2$\\
Somit:  $y' = f(x) \cdot y + g(x)$ mit $g(x) := 2x^2$ und $f(x) :=  - x^2$\\
Es gilt somit:\[
F(x) = \int f(t) dt. =  \int - x^2 dt. = -\frac{x^3}{3}
\]
Dann ist \[
\{y_{Hom}(x) = c_1e^{-\frac{x^3}{3}}| c_1 \in \R\}
\] die Menge aller Lösungen der homogenen Differnentialgleichung ($y' + x^2\cdot y = 0$). \\

\fbox{%
        \parbox{1\linewidth}{%
\textit{Dieser Teil muss nicht berechnet werden (Herleitung).}\\
Als Ansatz für die Lösung des inhomogenen Problems setze man $y_p(x) = u(x)e^{F(x)}$.
\\
Ansatzfunktion ist \[
y_p'(x) = u(x)f(x)e^{F(x)} + u'(x)e^{F(x)} = y(x)f(x) + u'(x)e^{F(x)}
\]
Also löst $y$ die inhomogene Differntialgleichung \[
y_p'(x) = y(x)f(x) + g(x)
\]
genau dann, wenn\[
u'(x) = y(x)f(x) + g(x)
\]
gilt.
        }%
}

Also folgt\[
u(x) = \int  g(x)e^{-F(x)} dx. =  2 \int  x^2e^{\frac{x^3}{3}} dx. = 2e^{\frac{x^3}{3}}
\]
Somit ist die Lösungsmenge von $y_p$
\[
\{ y_p'(x) = e^{-\frac{x^3}{3}} (2e^{\frac{x^3}{3}} + c_2) | c_2 \in \R \}
= \{ y_p'(x) =2 + c_3) | c_3 \in \R \}
\]
Die Lösungsmenge der generellen Lösung der allgemeinen Lösung ist somit $y$\[
\{ y(x) =  c_1e^{-\frac{x^3}{3}} +  2  | c_1 \in \R \}
\]

\subsubsection{Beispiel genereller Ansatz}
Gegeben: $y' + x^2 \cdot y = 2x^2$

Homogene DGL lösen: $y' + x^2 \cdot y = 0$
\begin{align*}
y' + x^2 \cdot y &= 0\\
\frac{dy}{dx} + x^2 \cdot y &= 0 \quad | -(x^2 \cdot y)\\
\frac{1}{dx}\, dy &= -x^2 \cdot y \quad | \div y\\
\frac{1}{dx} \frac{1}{y} \, dy &= -x^2 \quad | \int\\
\int \frac{1}{dx} \frac{1}{y} \, dy \, dx &= \int -x^2 \, dx\\
\int \frac{1}{y}\, dy &= \int -x^2 \, dx\\
\ln(y) &= -\frac{1}{3} x^3 \quad | e^\alpha\\
y &= e^{-\frac{1}{3}x^3}
\end{align*}

Somit ist die allgemeine homogene Lösung: $\underline{y_h = A \cdot e^{-\frac{1}{3}x^3}}$.


Als nächstes gehen wir die praktikuläre Lösung an:
$y_p = u(x) \cdot e^{-\frac{1}{3}x^3}$
\begin{align*}
\Rightarrow (u(x) \cdot e^{-\frac{1}{3}x^3})' + x^2 (u(x) e^{-\frac{1}{3}x^3}) &= 2 x^2\\
u'(x) \cdot e^{-\frac{1}{3}x^3} - u(x) \cdot x^2 e^{-\frac{1}{3}x^3} + u(x) x^2 e^{-\frac{1}{3}x^3} &= 2 x^2\\
u'(x) \cdot e^{-\frac{1}{3}x^3} &= 2 x^2 \quad | \div e^{-\frac{1}{3}x^3}\\
u'(x) &= 2 x^2 e^{\frac{1}{3}x^3} \quad | \int\\
u(x) &= 2 e^{\frac{1}{3}x^3}
\end{align*}

Wir erhalten somit: $\underline{y_p} = 2 e^{\frac{1}{3}x^3} \cdot e^{-\frac{1}{3}x^3} = \underline{2}$.
Die allgemeine Lösung des inhomogenen DGL ist somit:
$\underline{\underline{y}} = y_h + y_p = \underline{\underline{A \cdot e^{-\frac{1}{3}x^3} + 2}}$


\subsubsection{Beispiel Direkterer Lösungsweg}
Gegeben: $y' + x^2 \cdot y = 2x^2$.
Direkt lösen:
\begin{equation*}
\begin{array}{r l |l}
y' + x^2 \cdot y &= 2x^2\\
\frac{dy}{dx} + x^2 \cdot y &= 2x^2 \quad & -(x^2 \cdot y)\\
\frac{dy}{dx} &= 2x^2 -x^2 \cdot y \quad & \text{vereinfachen}\\
\frac{dy}{dx} &= x^2 ( 2 - y) \quad & \div (2 - y) \\
\frac{\frac{dy}{dx}}{2 - y} &= x^2  & \int \text{ with respect to } x \\
\int \frac{\frac{dy}{dx}}{2 - y}  \, dx &= \int x^2 \, dx & \text{links: } \int \frac{g'(x)}{g(x)} \, dx = \ln|g(x)|\\ 
- ln |2 - y| &= \frac{x^3}{3} + c_1 & \cdot (-1) \\
ln |2 - y| &= - \frac{x^3}{3} - c_1 & e \\
e^{ln |2 - y|} &= e^{-\frac{1}{3}x^3 - c_1 } \\
2 - y &= e^{-\frac{1}{3}x^3 - c_1 }& - 2 \\
- y &= e^{-\frac{1}{3}x^3 - c_1 } - 2 & \cdot (-1)\\
y &= - e^{-\frac{1}{3}x^3 - c_1 } + 2 & \text{replace const} \\
\underline{\underline{y}} &= \underline{\underline{c_2 \cdot e^{-\frac{1}{3}x^3} + 2}}\\
\end{array} 
\end{equation*}


\subsection{Lineare DGL n-ter Ordnung {\footnotesize mit konst. Koeffizienten}}
Diese DGL haben genau n Nullstellen und die Form:
\begin{eqnarray*}
	y^{(n)}+a_{n-1} \cdot y^{(n-1)}+\ldots+a_1 \cdot y' +a_0 \cdot y=g(x)\\
	\text{wobei} \; g(x) = 0 \hspace{10pt} \text{oder} \hspace{10pt} g(x) \neq 0
\end{eqnarray*}

\textbf{Vorgehen im homogenen Fall:}
\begin{enumerate}
	\item Homogene DGL aufstellen und dazu das charakteristische Polynom $p(\lambda)$ notieren mit Ansatz $y(x) = e^{\lambda x}$ mit $\lambda \in \C$:
	\begin{align*}
		y^{(n)} &+ a_{n-1} \cdot y^{(n-1)} &+ \ldots &+ a_1 \cdot y' &+& \, a_0 \cdot y&=0\\
		p(\lambda) = (\lambda^n &+ a_{n-1} \cdot \lambda^{n-1} &+ \ldots &+ a_1 \cdot \lambda &+& \, a_0) \cdot e^{\lambda x} &\overset{!}{=} 0 \\
		= \lambda^n &+ a_{n-1} \cdot \lambda^{n-1} &+ \ldots &+ a_1 \cdot \lambda &+& \, a_0 &\overset{!}{=} 0
	\end{align*}
	Merke: $e^{\lambda x}$ kann nie 0 sein, deshalb muss (...) = 0 sein!\\

	\item Nun müssen die ($\lambda_1, \ldots, \lambda_n$) Nullstellen von $p(\lambda)$ berechnet werden. Wenn $n > 2$ muss zuerst $p(\lambda)$ in lineare und quadratische Faktoren (durch raten/x-ausfaktorisieren/Polynomdivision/binomische Formeln) zerlegt werden. Dh: Jeder Faktor ist dann von 1. oder 2. Ordnung und davon können nun die Nullstellen berechnet werden (ablesen/Mitternachtsformel). Wir beachten, dass es sowohl reelle als auch komplexe Nullstellen gibt und merken für jede Nullstelle die Vielfachheit dieser Nullstelle.\\
	Bsp. einer linearer und quadratischen Zerlegung:
	{\small \begin{align*}
			p_1(\lambda) &= \lambda^2 + \lambda - 6 = (\lambda + 3) (\lambda - 2) \hspace{10pt} \text{oder} \\
			p_2(\lambda) &= \lambda^4 -4 = (\lambda^2 + 2) (\lambda^2 - 2) = (\lambda^2 + 2) (\lambda + \sqrt{2}) (\lambda - \sqrt{2})
	\end{align*}}
	\item 
	\begin{enumerate}[leftmargin=0.3cm]
		\item Ist $\lambda_i$ eine k-fache reelle Nullstelle, so gibt es k linear unabhängige Lösungen zur Nullstelle $\lambda_i$, nämlich:
		\vspace{3pt}
		\begin{align*}
			e^{\lambda_i x}, \, x e^{\lambda_i x}, \, x^2 e^{\lambda_i x}, \, ... , \, x^{k-1} e^{\lambda_i x}
		\end{align*}

		\item Sind $\lambda_i = a \pm ib$ k-fache komplexe Nullstellen, so gibt es 2k linear unabhängige Lösungen zu diesen 2 Nullstellen, nämlich:
		{\small\begin{align*}
				e^{\lambda_i x}, x e^{\lambda_i x}, \ldots, x^{k-1} e^{\lambda_i x}
				= \; &e^{x(a+ib)}, x e^{x(a+ib)}, \ldots, x^{k-1} e^{x(a+ib)}, \\
				&e^{x(a-ib)}, x e^{x(a-ib)} \ldots, x^{k-1} e^{x(a-ib)}
		\end{align*}}Merke: Durch anwenden der Eulersche Identität lässt sich obige komplexe Lösung auch reell schreiben als:
		\vspace{3pt}
		\begin{align*}
			&e^{a x} \, \cos(bx), \, x e^{a x} \, \cos(bx), \ldots, \, x^{k-1} e^{a x} \, \cos(bx) \hspace{10pt} \text{und}\\
			&e^{a x} \, \sin(bx), \, x e^{a x} \, \sin(bx), \ldots, \, x^{k-1} e^{a x} \, \sin(bx)
		\end{align*}
	\end{enumerate}
	\vspace{3pt}
	\item Die allgemeine Lösung dieser homogenen DGL ist nun eine linearkombination all dieser gefundenen Lösungen. \\
	Bsp: $\lambda_1 =$ einfache, $\lambda_2 = $ 2-fache reelle Nullstelle:
	\begin{align*}
	 y_h(x) = C_1 e^{\lambda_1 x} + \ucomment{\lambda_2: \text{2-fache Nullstelle}}{C_2 e^{\lambda_2 x} + C_3 x e^{\lambda_2 x}}
	\end{align*}
	Die Unbekannten $C_i$ können gefunden werden, wenn genügend Punkte gegeben sind, an denen der Funktionswert bekannt ist. Einfach $y$ an den gegebenen Punkte auswerten und Gleichung mit dem bekannten Resultat aufstellen.
\end{enumerate}

\textbf{\textbf{Vorgehen im inhomogenen Fall:}}
\begin{enumerate}
	\item Zuerst die homogene Lösung finden (siehe oben)!

	\item Zuerst wähl man einen geeigneten Ansatz für das $y_p$ siehe \ref{sec:ansatz-dgl})

	\item Hat man einen allgmeinen Ansatz für $y_p$ mit noch unbekannten konstanten bestummen, so werden jetzt die nächsten $n$ Ableitung davon berechnet. ($n$ = Ordnung der DGL). 

	\item Nun setzt man die berechneten Ableitungen in die ursprüngliche inhomogene DGL ein. Dh: ersetze $y$ mit $y_p$, $y'$ mit der 1. Ableitung von $y_p$ etc. Wenn möglich Terme vereinfachen.

	\item Jetzt die unbekannten Konstanten bestimmen indem man einen Koeffizientenvergleich macht. Dh: Gleichungen aufstellen, so das linke Seite der DGL der rechten Seite entspricht. 

	\item Gefundene Konstanten können jetzt in die partikuläre Lösunge eingesetzt werden.\\
	Die allgemeine Form der DGL ist $y(t) = y_h(t) + y_p(t)$
\end{enumerate}

Hat man ursprünglich eine inhomogene DGL vorliegen, so muss man für die allgemeine Lösung
noch die partikuläre Lösung des inhomogenen DGL berechnen. Dazu werden die Unbekannten
$C_i$ durch Funktionen $u_i(x)$ ersetzt. So wird aus
$y_h = C_2 x e^{\lambda_1 x} + C_1 e^{\lambda_1 x} \Rightarrow
y_p = u_2(x) x e^{\lambda_1 x} + u_1(x) e^{\lambda_1 x}$ (eine doppelte Nullstelle).

Jetzt geht es darum die Funktionen $u_i(x)$ zu bestimmen, um sie in die vorherige
$y_p$-Gleichung einsetzen zu können. Dazu stellen wir $i$ Gleichungen auf.
Also so viele, wie wir unbekannte Funktionen $u_i(x)$ haben:
\begin{align*}
u_2(x)' (x e^{\lambda_1 x}) + u_1(x)' (e^{\lambda_1 x}) &= 0\\
u_2(x)' (x e^{\lambda_1 x})' + u_1(x)' (e^{\lambda_1 x})' &= g(x)
\end{align*}

Das Prinzip ist folgendes: Bis auf die letzte Gleichung, wird gleich $0$ gesetzt.
Die letzte Gleichung wird gleich $g(x)$ gesetzt.
Unsere unbekannten Funktionen werden jeweils einmal abgeleitet, egal in welcher
Gleichung wir sind. Pro Zeile, die man weiter runter geht, wird der Term mit $e^{\lambda_i x}$
jeweils einmal mehr abgeleitet. In der ersten Zeile wird zum Beispiel $e^{\lambda_1 x}$ nicht abgeleitet,
in der nächsten Gleichung wird es einmal abgeleitet. Hätten wir mehr Unbekannte Funktionen,
so würde in der folgenden Zeile zwei mal abgeleitet werden. Im Allgemeinen gilt also:
{\footnotesize
\begin{align*}
u_1(x)' y_{h1}(x) + u_2(x)' y_{h2}(x) + \ldots + u_n(x)' y_{hn}(x) &= 0\\
u_1(x)' y_{h1}(x)' + u_2(x)' y_{h2}(x)' + \ldots + u_n(x)' y_{hn}(x)' &= 0\\
u_1(x)' y_{h1}(x)'' + u_2(x)' y_{h2}(x)'' + \ldots + u_n(x)' y_{hn}(x)'' &= 0\\
&\ldots\\
u_1(x)' y_{h1}(x)^{(n-1)} + u_2(x)' y_{h2}(x)^{(n-1)} + \ldots + u_n(x)' y_{hn}(x)^{(n-1)} &= g(x)
\end{align*}
}

Diese Gleichungen werden nun jeweils aufgelöst, bis man $u_i(x)$ erhält. Um zu
$u_i(x)$ zu gelangen, muss auf dem Weg einmal die Gleichung auf beiden Seiten
integriert werden. Hat man alle $u_i(x)$, so setzt man diese in unsere
ursprüngliche $y_p$ Gleichung ein.

Nun kann die allgemeine Lösung des inhomogenen DGL berechnet werden. Dazu
summiert man $y_h$ und $y_p$: $y = y_h + y_p$. Dies ist die allgemeine Lösung.
Hat man konkrete Punkte, an denen die Funktion ausgewertet wurde, so kann man
die Unbekannten $C_i$ berechnen.

\subsubsection*{Beispiel}
Es soll $y'' + y = \frac{2}{\cos(x)}$ ausgerechnet werden.

Zuerst sehen wir uns das homogene DGL an:
\begin{align*}
y'' + y &= 0\\
\Rightarrow \lambda^2 e^{\lambda x} + e^{\lambda x} &= 0\\
\Leftrightarrow e^{\lambda x} (\lambda^2 + 1) &=0 \\
\Rightarrow \lambda^2 + 1 &= 0\\
\Leftrightarrow \lambda^2 &= -1 \quad
\Rightarrow \underline{\lambda_1 = i},\, \underline{\lambda_2 = -i}
\end{align*}

Somit ist die allgemeine Lösung des homogenen DGL:
\begin{align*}
y_h &= C_1 \cdot e^{ix} + C_2 \cdot e^{-ix}\\
&= C_1 (\cos(x) + i\sin(x)) + C_2(\cos(x) - i\sin(x))\\
&= \sin(x) \underbrace{(i C_1 + i C_2)}_{ = D_1} + \cos(x) \underbrace{(C_1 + C_2)}_{= D_2}\\
&= \underline{D_1 \sin(x) + D_2 \cos(x) = y_h}
\end{align*}

Da es sich um ein inhomogenes DGL handelt, berechnen wir als nächstes die partikuläre
Lösung:
\begin{align*}
y_p &= \underbrace{u_1(x)}_{D_1 \text{ in } y_h} \sin(x) + \underbrace{u_2(x)}_{D_2 \text{ in } y_h} \cos(x)\\
&\Rightarrow \left|
	\begin{aligned}
		u_1(x)' \sin(x) + u_2(x)' \cos(x) &= 0\\
		u_1(x)' \sin(x)' + u_2(x)' \cos(x)' &= \frac{2}{\cos(x)}
	\end{aligned}
\right|\\
&= \ldots\\
&\Rightarrow u_1(x)' = -2 \tan(x),\, u_2(x)' = 2\\
&\Leftrightarrow u_1(x) = 2 \ln(\cos(x)),\, u_2(x) = 2x\\
&\Rightarrow \underline{y_p = 2 \ln(\cos(x)) \sin(x) + 2x \cos(x)}
\end{align*}

Da wir nun auch die partikuläre Lösung haben, können wir die allgemeine Lösung
des inhomogenen DGL berechnen:
$\underline{\underline{y}} = y_h + y_p = \underline{\underline{D_1 \sin(x) + D_2 \cos(x) + 2 \ln(\cos(x)) \sin(x) + 2x \cos(x)}}$

\subsection{Ansätze für partikuläre Lösung}
\label{sec:ansatz-dgl}
	\textbf{Hinweis}:
	\begin{itemize}
		\item Ansätze nur brauchbar für lineare DGL mit konstanten Koeffizienten.

		\item Die gesuchte Funktion $y$ ist immer vom gleichen Grad wie die Störfunktion $q(x)$.

		\item Wenn $q(x)$ eine Linearkombination von Funktionen ist, so muss man auch einen entsprechenden Ansatz wählen! Dh: Für jeden Summanden von $q(x)$ einzeln eine partikuläre Lösung finden und am Ende addieren!
	\end{itemize}
	Bezeichnungen:
	\begin{align*}
	P(x)  &  \quad \text{charakt. Polynom der DGL}  \\
	S_k(x) & \quad \text{polynomielle Störfunktion, Grad} \; k \\
	A, B & \quad \text{unbekannte Konstanten} \\
	R_k(x) = a_k x^k + . + a_1 x + a_0 & \quad \text{mit unbekannten Koeffizienten}
	\end{align*}
	
	
	
	\begin{tabular}{l|l}
		$q(x)$ & $Ansatz$ \\ \hline \hline
		
		$ S_k(x) $ & $R_k(x)$ , falls $P(0) \neq 0$\\
		\small{$S_1(x): ax + b$} &  $x^q R_k(x)$, falls $0$ $q$-fache NST von $P$ \\ 
		\small{$S_2(x): ax^2 + bx + c$} & \\ \hline

		$ c e^{mx} $  & $A e^{mx}$ , falls $P(m) \neq 0$\\ 
		&  $A x^q e^{mx}$, falls $m$ $q$-fache NST von $P$ \\ \hline

		$ S_k(x) e^{mx} $  & $R_k(x) e^{mx}$ , falls $P(m) \neq 0$\\
		&  $x^q R_k(x) e^{mx}$, falls $m$ $q$-fache NST von $P$ \\ \hline
		$ \sin wx, \cos wx $  & $A \cos wx + B \sin wx$ , falls $P(\pm iw) \neq 0$\\
		&  $x^q (A \cos wx + B \sin wx)$, falls $\pm iw$ \\
		& $q$-fache NST von $P$ \\ \hline
		$ \sinh wx, \cosh wx $  & $A \cosh wx + B \sinh wx$ , falls $P(w) \neq 0$\\
		&  $x^q (A \cosh wx + B \sinh wx)$, falls $w$ \\
		& $q$-fache NST von $P$ \\
	\end{tabular}
