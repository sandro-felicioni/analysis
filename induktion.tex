\section{Vollständige Induktion}\index{Induktion}
Grundlegende Struktur um die Aussage $A(n)$ zu beweisen:
\begin{enumerate}
	\item \textbf{Verankerung/Induktionsanfang:} Die Aussage wird für $n = A$ bewiesen.
	$A$ ist dabei meistens der erste Wert für die gegebene Eingabemenge.
	Der Beweis wird meist durch direktes ausrechnen gemacht.
	\item \textbf{Annahme/Induktionsvoraussetzung:} Hier schreibt man,
	dass man davon ausgeht die Aussage sei gültig für ein bestimmtes $n \in \N$ (damit man sie im 4. Schritt
	einsetzen kann). Man kopiert also im Grunde, was man zu beweisen hat mit einigen Zierwörter.
	\item \textbf{Induktionsbehauptung:} Hier schreibt man, dass die Aussage auch für (n + 1) gilt.
	\item \textbf{Induktionsschritt:} Hier beweist man, dass unter der Annahme, dass die Induktionsvoraussetzung gilt, die Induktionsbehauptung folgt. Oder anders gesagt, wir beweisen dass wenn die Aussage für n gilt, dass es dann auch für (n + 1) gelten muss. Dazu wird die Induktionsannahme verwendet.
\end{enumerate}
Merke: Schritt 2 und 3 werden oft weggelassen, wenn zu trivial!\\
\textbf{Beispiel}\\
Es ist zu beweisen, dass für jedes $n \in \N$ folgendes gilt:\\
$1 + 2 + 3 + \ldots + n = \frac{n(n + 1)}{2}$
\begin{enumerate}
	\item \textbf{Verankerung:} Für $n = 1$ gilt: $1 = \frac{1 (1 + 1)}{2} = \frac{2}{2} = 1 \quad \checkmark$ 
	\item \textbf{Annahme:} $1 + 2 + 3 + \ldots + n = \frac{n(n+1)}{2}, \; \text{für ein} \; n \in \N$
	\item \textbf{Induktionsbehauptung:} Wenn die Aussage für n gilt, dann gilt sie auch für (n + 1).
	\item \textbf{Induktionsschritt:} Für $n \to n+1$ gilt:
	\begin{align*}
	1 + 2 + \ldots + n + (n+1) \overset{\text{Annahme}}{=} & \frac{n(n+1)}{2} + (n + 1)\\
	&= \frac{n(n+1)}{2} + \frac{2n + 2}{2} \\
	&= \frac{n^2 + n + 2n + 2}{2} \\
	&= \frac{(n + 1)(n + 2)}{2} _\square
	\end{align*}
\end{enumerate}