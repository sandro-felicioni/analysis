\section{Funktionsfolgen}
Wenn $f_n : \Omega \to \R$ eine Funktion ist (für jedes $n \in \N$) dann nennt man $f_n$ eine Funktionenfolge. Dh: Eine Folge von Funktionen.

\begin{definition}[punktweise konvergent] \index{punktweise konvergent}
Die Funktionenfolge $f_n$ \underline{konvergiert punktweise} gegen $f$, falls gilt:
\[
	\lim_{n \to \infty} f_n(x) = f(x) \hspace{1cm} \text{für} \; \forall x \in \Omega
\]
Formal: 
\[
\forall \epsilon > 0 \; \forall x \in \Omega \; \exists n_0 \in \N: n \geq n_0
\Rightarrow |f(x) - f_n(x)| < \epsilon
\]
\end{definition}


\begin{definition}[gleichmässig konvergent] \index{gleichmässig konvergent}
Die Funktionenfolge $f_n$ \underline{konvergiert gleichmässig} gegen $f$, falls gilt:
\[
	\lim_{n \to \infty} \sup_{x \in \Omega} |f_n(x) - f(x)| = 0 
\]
Formal: 
\[
\forall \epsilon > 0 \; \exists n_0 \in \N \; \forall x \in \omega: n \geq n_0
\Rightarrow |f(x) - f_n(x)| < \epsilon
\]
Das bedeutet, dass die obere Definition für alle $x$ \underline{dasselbe} $n_0$ verwendet und nicht jeweils
verschiedene!
\end{definition}

\underline{Wichtig}: gleichmässige Konvergenz $\Rightarrow$ punktweise
Konvergenz.